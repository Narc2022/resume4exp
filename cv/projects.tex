%-------------------------------------------------------------------------------
%	SECTION TITLE
%-------------------------------------------------------------------------------
\cvsection{Projects}


%-------------------------------------------------------------------------------
%	CONTENT
%-------------------------------------------------------------------------------


%----------------------------------------
\begin{cventries}


%---------------------------------------------------------

\cventry
  {Front-End Development} % Organisation
  {Component Library (Internal)} % Project
  {Multiple Dates} % Date(s)
  {} % Location
  {
    \begin{cvitems} % Description(s) of project
      \item {Established a robust foundation combining \textbf{Storybook} and \textbf{React}.}
      \item {Engineered a seamless mechanism for sharing components across all frontend platforms.}
      \item {Significantly reduced development efforts, accelerating project timelines.}
      \item {Fostered the creation of optimized and reusable code snippets.}
    \end{cvitems}
}

\cventry
  {Front-End Development} % Organisation
  {Compass Platform (B2C)} % Project
  {Multiple Dates} % Date(s)
  {} % Location
  {
    \begin{cvitems} % Description(s) of project
      \item {Employed \textbf{React.js} for a responsive and visually appealing user interface.}
      \item {Developed a modular architecture for learning paths, utilizing \textbf{React Router} for seamless navigation.}
    \end{cvitems}
}

\cventry
  {Front-End Development} % Organisation
  {Azent Website Scholarship Page (B2C)} % Project
  {Multiple Dates} % Date(s)
  {} % Location
  {
    \begin{cvitems} % Description(s) of project
      \item {Engineered custom components for seamless integration.}
      \item {Collaborated with design for user-centric visuals.}
      \item {Ensured mobile responsiveness for optimal viewing.}
      \item {Integrated front-end with backend for data consistency.}
    \end{cvitems}
}

%---------------------------------------------------------
\cventrynew
  {Portfolio} % Organisation
  {01/2022 - 06/2022 } % Project
  {Full Stack Development} % Date(s)
  {} % Location
  {
    \begin{cvitems}
      \item {Tech Stack: {\bf MERN (MongoDB, Express.js, React.js, Node.js)}}
      \item {Link: \href{https://cluster-front.vercel.app/}{https://cluster-front.vercel.app/}}
      \item {Created a portfolio website featuring a Content Management System (CMS) for managing timelines and projects.}
      \item {Allows easy updates and showcases all projects and professional timelines in a structured and visually appealing manner.}
    \end{cvitems}
  }


% %---------------------------------------------------------
%   \cventry
%     {Industry 4.0 and Advanced Manufacturing: Proceedings of I-4 AM 2022 (Published paper)} % Organisation
%     {Design of autonomous carrier robot in Industrial Applications } % Project
%     {Bangalore, India} % Location
%     {06/2022 - 07/2022} % Date(s)
%     {
%       \begin{cvitems} % Description(s) of project
%         \item {Tasks incorporated in General Hospital U.T. of Puducherry, India, to perform transport operations with possible non-static obstacles in the path.}
%         \item {Used the shortest possible time for the given operations through a pre-defined map.}
%       \end{cvitems}
    % }

%---------------------------------------------------------
  % \cventry
  %   {e-Yantra, International Robotics Competition, Indian Institute of Technology Bombay.} % Organisation
  %   {Autonomous Robot | Virtual \& Real-life Simulations (Won Competition Finalists)} % Project
  %   { Oct 2019 - Feb 2020} % Date(s)
  %   {} % Location
  %   {
  %     \begin{cvitems} % Description(s) of project
  %        \item {Built a robot from scratch possessing vision, picking, placing, and autonomous decision-making capabilities.}
  %       \item {Worked with {\bf 2D Path Planning}(A* \& Dijkstra) algorithms to take the shortest path during natural emergencies.}
  %       %\item {Awarded as one of "India's National Finalists." Senior academic professors appreciated for creating a {\bf robust robotic prototype} model.}
  %     \end{cvitems}
  %   }

% %---------------------------------------------------------
%   \cventry
%     {International Conference of Mechanical Engineering, Netaji Subhas University of Technology (Published)} % Organisation
%     {    \href{https://docs.google.com/presentation/d/15VEjcrCeRimSCmY5w_ymFGQUNm8xMUAd83Tfsv_nm_8/edit?usp=sharing}{\underline{\textcolor{blue}{Energy efficient coatings for improved wear and tear performance of a CWP}}}} % Project
%     {New Delhi, India} % Location
%     {09/2021 - 10/2021} % Date(s)
%     {
%       \begin{cvitems} % Description(s) of project
%         \item {Polymer Coatings on hydrophobic passages have increased the efficiency of CWP by 5\%. And also protects from wear and tear, UV, high temperature, corrosion, and erosion.}
%       \end{cvitems}
%     }
    
% %---------------------------------------------------------
%   \cventry
%     {Indian Institute of Technology Patna  (Bachelor project)} % Organisation
%     {
%     \href{https://docs.google.com/presentation/d/1IUcf8SZGhImtNFdskbcalEr65jUCI7tMqxSCZA2hNxg/edit?usp=sharing}{\underline{\textcolor{blue}{Visualization of 2-D motion of a towed floating object}}}} % Project
%     {Bihar, India} % Location
%     {07/2021 - 08/2022} % Date(s)
%     {
%       \begin{cvitems} % Description(s) of project
%         \item {To create a simulator user-interface module to visualize a towed floating object based on {\bf Kinematic and Dynamic model}.}
%         \item {\textbf{Technical Skills:} Matlab, matplotlib, Python in GlowScript.}
%       \end{cvitems}
%     }

%---------------------------------------------------------
\end{cventries}
